%%%  This is a set of styles to include in the preamble of the generated latex code for the book Discrete Mathematics: an Open Introduction.


\usepackage{bold-extra}
\usepackage{marvosym} %for stop signs.
\usepackage{textcomp}
\usepackage{multicol}


% % FONT OPTIONS (pick one group to uncomment):

%newpx is my current favorite.  This should work on a TEXLive distribution, but on MiKTeX it initially gave me problems.  Running the update wizard to update MiKTeX then showed ``newpx'' as an available package (only since 8/11/15).  Perhaps just synchronizing in the package manager would do the same thing.  Even then, needed to run initexmf --mkmaps from the command line.
%You could always uncomment all of these to use the default computer modern font.
\usepackage[utf8]{inputenc}
\usepackage[T1]{fontenc}
\usepackage{newpxtext}
\usepackage[vvarbb,cmintegrals,cmbraces,bigdelims]{newpxmath}
\usepackage[scr=rsfso]{mathalfa}% \mathscr is fancier than \mathcal
\linespread{1.04}         % adds more leading (space between lines)
% quantifiers look strange, so change those back to normal:
	\DeclareSymbolFont{mysymbols}{OMS}{cmsy}{b}{n} %note we make the figures bold to better match newpx.  Replace the ``b'' with an ``m'' to undo this.
	%\SetSymbolFont{mysymbols}  {bold}{OMS}{cmsy}{b}{n}
	%\DeclareSymbolFont{myoperators}   {OT1}{cmr} {m}{n}
	%\SetSymbolFont{myoperators}{bold}{OT1}{cmr} {bx}{n}
	\DeclareMathSymbol{\forall}{\mathord}{mysymbols}{"38}
	\DeclareMathSymbol{\exists}{\mathord}{mysymbols}{"39}
	%\DeclareMathSymbol{\pm}{\mathbin}{mysymbols}{"06}
	%\DeclareMathSymbol{+}{\mathbin}{myoperators}{"2B}
	%\DeclareMathSymbol{-}{\mathbin}{mysymbols}{"00}
	%\DeclareMathSymbol{=}{\mathrel}{myoperators}{"3D}



%\usepackage[T1]{fontenc}
%\usepackage{newtxtext,newtxmath}


%\usepackage[bitstream-charter]{mathdesign}
%\usepackage[T1]{fontenc}

%\usepackage[proportional,space,scaled=1.064]{erewhon}
%\usepackage[erewhon,vvarbb,bigdelims]{newtxmath}
%\usepackage[T1]{fontenc}
%\renewcommand*\oldstylenums[1]{\textosf{#1}}



% 
% 
% % % % % % % Other packages % % % % % % % %
% 
% \usepackage{docmute}
% \usepackage{pdfpages}
% 
% \usepackage{svg}
% 
% \usepackage[framemethod=tikz]{mdframed}
% 
% 
% 
% % % % % % % % % % % % % % %  END OF PACKAGES  % % % % % % % % % % % % % % % % % % %
% 
% %%%%%%%%%%%%%%%%%%%%%%%%%%%%%%%%%%%%%%%%%%%%%%%%%%%%%%%%%%%%%%%%%%%%%%%%
% %%%%%%%%%%%%%%%%%  Nicely styled environments: %%%%%%%%%%%%%%%%%%%%%%%%%
% %%%%%%%%%%%%%%%%%%%%%%%%%%%%%%%%%%%%%%%%%%%%%%%%%%%%%%%%%%%%%%%%%%%%%%%%
% 
% Tweak exercises styling


\renewtcolorbox{divisionexercise}[4]{bwminimalstyle, runintitlestyle, exercisespacingstyle, left=4ex, breakable, parbox=false, before title={\hspace{-4ex}\makebox[4ex][l]{#1.}}, title={\notblank{#2}{#2\space}{}}, phantom={\hypertarget{#4}{}},  after={\notblank{#3}{\newline\rule{\workspacestrutwidth}{#3\textheight}\newline}{\vskip 1em}}}

\setlist{itemsep=2pt plus 1pt minus 1pt}

\reversemarginpar



% %%%%%%%%%%%%%%%%%  End environments %%%%%%%%%%%%%%%%%%%%%%%%%%
% 
% Start sections on a new page.  Add \clearpage to the second command to start subsections on their own page.
%\newcommand{\sectionbreak}{\ifnum\value{section}>1\clearpage\fi\phantomsection}
%\newcommand{\subsectionbreak}{\phantomsection}
% 
% % use QED to end proofs:
%\renewcommand{\qedsymbol}{\textsc{qed}}
% 
% 
% %%%%%%%% Set description list spacing the way I want %%%%%%%%%%%%%%%

\setlist[description]{style=nextline, leftmargin=3em}
% 
% 
% %%%%%%% Set chapters to start at 0 %%%%%%%%%%%%%%%%%%%%%%%%%%%
%\setcounter{chapter}{0}
% 
% 
% %Fix widows and orphans (single lines at top/bottom of page):
\clubpenalty=10000
\widowpenalty=10000
\raggedbottom


%%% End of File.